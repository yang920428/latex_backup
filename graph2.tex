% !TEX root = graph2.tex

\documentclass[twocolumn]{article}
\usepackage{graphicx} % Required for inserting images
\usepackage{CJKutf8}
% \usepackage{amsthm}
\usepackage{mdframed}
\usepackage{multicol} % 用於分欄
\usepackage{geometry} % 用於調整頁面尺寸和邊距
\usepackage{amsmath}
\usepackage{titlesec}

\newtheorem{example}{example}             % 整体编号
\newtheorem{algorithm}{algorithm}
\newtheorem{theorem}{theorem}[section]  % 按 section 编号
\newtheorem{definition}{definition}
\newtheorem{axiom}{axiom}
\newtheorem{property}{property}
\newtheorem{proposition}{proposition}
\newtheorem{lemma}{lemma}
\newtheorem{corollary}{corollary}
\newtheorem{remark}{remark}
\newtheorem{condition}{condition}
\newtheorem{conclusion}{conclusion}
\newtheorem{assumption}{assumption}
\newtheorem{formula}{formula}
\newtheorem{problem}{problem}
\newtheorem{solution}{solution}

\linespread{0.5}
% \setlength{\parskip}{0.001em}

\setlength{\topsep}{0.1em} % 環境之上和之下的距離
\setlength{\partopsep}{0pt} % 環境和上下文之間的額外距離
\setlength{\itemsep}{0.5em} % 項目之間的距離
\setlength{\parsep}{0pt} % 段落之間的距離

\titlespacing*{\section}{0pt}{0pt}{0pt}

% 設定頁面尺寸和邊距
\geometry{
    top=0.5cm,
    bottom=0.5cm,
    left=0.5cm,
    right=0.5cm
}


% 自定義 "Definition" 環境
% \newmdtheoremenv{definition}{Definition}
% \theoremstyle{mystyle}

\title{}
% \author{}
% \date{}

\begin{document}
\begin{CJK*}{UTF8}{bkai}

% \maketitle

% 第一部分(Chapter 1)
\section{Fundamentals}
    \begin{definition}
         A cut-edge or cut-vertex of a graph is an edge or vertex whose deletion
 increases the number of components.
    \end{definition}
% \vspace{2cm} % 分隔不同章節

% 第二部分(Chapter 2)
\section{Tree}
% \begin{multicol}{2} % 第二部分分欄
    % \switchcolumn[0]

    \begin{theorem}{(Cayley's Formula,)}
         For a set S $\subseteq$ N of size n, there are $n^{n-2}$ trees with vertex set S.
    \end{theorem}

% 第三部分(Chapter 3)
\section{Matching and factor}
    \begin{definition}
        A matching in a graph G is a set of non-loop edges with no shared endpoints.
 The vertices incident to the edges of a matching M are saturated by M or
 M-saturated; the others are M-unsaturated. A perfect matching in a graph is
 a matching that saturates every vertex. The size of a matching is the number
 of edges in it.
    \end{definition}

\section{HW}

\begin{problem}
 Two people play a game on a graph G, alternately choosing distinct vertices. Player 1 starts by
 choosing any vertex. Each subsequent choice must be adjacent to the preceding choice (of the other
 player). Thus together they follow a path. The last player able to move wins.
 Prove that Player 2 has a winning strategy if G has a perfect matching, and otherwise Player 1 has
 a winning strategy.
\end{problem}

\begin{solution}
 If G has a perfect matching M, then Player 2 cannot lose if they always chooses the vertex paired
 with the preceding choice of Player 1. There is no draw in the game because the graph is finite and
 eventually somebody has no vertex to choose. Thus Player 2 can win in this way.
 If G does not have a perfect matching, let M be a maximum matching of G. Player 1 starts by
 choosing a vertex u that is M-unsaturated. From now on, when it is Player 2's turn there is no
 M-unsaturated vertex for them to choose because otherwise all the vertices that have been chosen
 form an M-augmenting path, implying that M is not maximum. So Player 2 is always forced to
 choose an M-saturated vertex, and Player 1 cannot lose if they always chooses the vertex paired
 with the preceding choice of Player 2. Similarly, Player 1 wins because there is no draw.
\end{solution}

% \subsection{hw1}
%     \begin{problem}
%  Prove that no bipartite graph contains an odd cycle    
%     \end{problem}
    
%     \begin{solution}
%     Suppose that A, B-bigraph G contains an odd cycle C = (v1, v2, . . . , v 2k+1). Without loss of generality suppose that v1 $\in$ A. Since G is bipartite and v1 v2 $\in$ E(G), v2 $\in$ B. Similarly, v2 v3 $\in$ E(G)
% and thus v3 $\in$ A, and so on. So v2k+1 $\in$ A. However v1 v2k+1 $\in$ E(G) and yet v1, v 2k+1 $\in$ A, a
% contradiction to the hypothesis that G is an A, B-bigraph.

%     \end{solution}
    
% \subsection{hw2}
%     \begin{problem}
%         Given a simple graph G, $\bar{G}$ is the simple graph with vertex set V(G) such that for any distinct
%  vertices u,v $\in$ V (G), u,v are adjacent in $\bar{G}$ if and only if u,v are not adjacent in G.
%  Prove that at least one of G and $\bar{G}$ is connected.
%     \end{problem}

%     \begin{solution}
%         If G is connected then we are done, so suppose that G is disconnected and has multiple components.
% We aim to prove that any pair of vertices x, y are connected in $\bar{G}$. If x, y belong to different
% components in G, then xy $\notin$ E(G) and xy $\in$ E($\bar{G}$). If x, y belong to the same component C in G,
% then fix a vertex v in a different component in G. Both x, y are adjacent to v in $\bar{G}$, thus x, y are
% connected in $\bar{G}$.

%     \end{solution}

% \subsection{hw3}

% \subsection{hw4}

% \subsection{hw5}

% \subsection{hw6}

\end{CJK*}
\end{document}
