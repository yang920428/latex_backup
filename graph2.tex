% !TEX root = graph2.tex

\documentclass[twocolumn]{article}
\usepackage{graphicx} % Required for inserting images
\usepackage{CJKutf8}
% \usepackage{amsthm}
\usepackage{mdframed}
\usepackage{multicol} % 用於分欄
\usepackage{geometry} % 用於調整頁面尺寸和邊距
\usepackage{amsmath}
\usepackage{titlesec}

\newtheorem{example}{example}             % 整体编号
\newtheorem{algorithm}{algorithm}
\newtheorem{theorem}{theorem}[section]  % 按 section 编号
\newtheorem{definition}{definition}
\newtheorem{axiom}{axiom}
\newtheorem{property}{property}
\newtheorem{proposition}{proposition}
\newtheorem{lemma}{lemma}
\newtheorem{corollary}{corollary}
\newtheorem{remark}{remark}
\newtheorem{condition}{condition}
\newtheorem{conclusion}{conclusion}
\newtheorem{assumption}{assumption}
\newtheorem{formula}{formula}
\newtheorem{problem}{problem}
\newtheorem{solution}{solution}
\newtheorem{fact}{fact}

\linespread{0.5}
% \setlength{\parskip}{0.001em}

\setlength{\topsep}{0.1em} % 環境之上和之下的距離
\setlength{\partopsep}{0pt} % 環境和上下文之間的額外距離
\setlength{\itemsep}{0.5em} % 項目之間的距離
\setlength{\parsep}{0pt} % 段落之間的距離

\titlespacing*{\section}{0pt}{0pt}{0pt}

% 設定頁面尺寸和邊距
\geometry{
    top=0.5cm,
    bottom=0.5cm,
    left=0.5cm,
    right=0.5cm
}


% 自定義 "Definition" 環境
% \newmdtheoremenv{definition}{Definition}
% \theoremstyle{mystyle}

\title{}
% \author{}
% \date{}

\begin{document}
\begin{CJK*}{UTF8}{bkai}

% \maketitle

% 第一部分(Chapter 1)
\section{matching}
    \begin{theorem}
     (Tutte 1947). A graph G has a 1-factor $\iff$ 
     o(G-S) $\leq$ |S| for every S $\subseteq $ V(G).
    \end{theorem}
% \vspace{2cm} % 分隔不同章節

    \begin{corollary}
       (Peterson 1891) Every 3-regular graph with no cut-edge has a 1-factor.  
    \end{corollary}

    \begin{definition}
        $deiect_{G} = \max _{S \subseteq V(G)} o(G-S)-|S|$
    \end{definition}

    \begin{theorem}
        (Berge-Tutte Formula, Berge 1958). The maximum number of vertices saturated by a matching in G is
 n(G) -defectG = min
 $S \subseteq V(G)$
 (n(G) -(o(G-S)-|S|))
    \end{theorem}

% 第二部分(Chapter 2)
\section{Connectivity}
% \begin{multicol}{2} % 第二部分分欄
    % \switchcolumn[0]

    \begin{definition}{}
         A separating set or vertex cut of a graph G is a set S ⊆ V(G) s.t. G−S has
 more than one component.
    \end{definition}

    \begin{definition}
         The connectivity of G, written $\kappa$(G), is the minimum size of a vertex set S s.t.
 G-S is disconnected or has only one vertex. A graph G is k-connected if
 $\kappa$(G) $\geq$ k, i.e., G-S is connected with at least two vertices for every
 k -1-vertex set S.
    \end{definition}

    \begin{lemma}
        $\kappa$(G) = n(G)-1 $\iff$ G contains Kn(G), i.e., every vertex is adjacent to every
 other vertex.
    \end{lemma}

    \begin{definition}
         A disconnecting set of edges is a set F $\subseteq $ E(G) s.t. G-F has more than one
 component.
    \end{definition}

    \begin{definition}
         A graph is k-edge-connected if every disconnecting set has at least k edges.
 The edge-connectivity of G, written $
 \kappa$'(G), is the minimum size of a
 disconnecting set or equivalently, the maximum k s.t. G is k-edge-connected
    \end{definition}

    \begin{fact}
    Every edge cut is a disconnecting set.
    \end{fact}

    \begin{fact}
         Every minimal disconnecting set of edges is an edge cut.
    \end{fact}

    \begin{definition}
        $\delta (G) = minimum vertex degree of G$
    \end{definition}

    \begin{theorem}{ (Whitney 1932).}
         $\kappa (G) \leq \kappa '(G) \leq \delta (G)$ for every graph G.
    \end{theorem}

    \begin{theorem}
         If G is a 3-regular graph of order more than 2, then $\
         kappa $(G) = $\kappa $'(G).
    \end{theorem}

    \begin{definition}
         Two paths are internally disjoint if they do not have common internal vertices.
         Two x,y-paths are edge-disjoint if they do not share any edges
        \end{definition}

    \begin{theorem}{(Whitney)}
         A graph G having at least 3 vertices is 2-connected $\iff$ for every distinct
 u,v $\in$ V(G) there exist internally disjoint u,v-paths in G.
    \end{theorem}

    \begin{lemma}{(Expansion Lemma)}
        If G is k-connected, and G' is obtained from G by adding a new vertex y with
 at least k neighbors in G, then G' is k-connected.
    \end{lemma}

    \begin{theorem}
         For a graph G of order at least 3, the following conditions are equivalent and
 characterize 2-connected graphs.

 \begin{itemize}
    \item G is connected and has no cut-vertex.
 \item For all x,y $\in$ V(G) there are internally disjoint x,y-paths.
 \item For all x,y $\in$ V(G) there is a cycle through x and y.
 \item $\delta$(G) $\geq$ 1 and every pair of edges in G lies on a common cycle.\end{itemize}
    \end{theorem}

    \begin{definition}
         A subdivision of an edge e = xy is a replacement of e with path x,z,y where z
 is a new vertex.
    \end{definition}

    \begin{corollary}
         If G is 2-connected, then the graph G' obtained by subdivision of an edge
 e =xy of G is 2-connected.
    \end{corollary}

    \begin{definition}
         An ear of a graph G is a nontrivial path in G whose endpoints have degree at
 least 3 and all its internal vertices have degree 2.
    \end{definition}

    \begin{definition}
         A decomposition of a graph is a list of subgraphs s.t. every edge appears in
 exactly one subgraph in the list.
    \end{definition}

    \begin{definition}
         An ear decomposition of G is a 
         decomposition P0,...,Pk s.t. 
         P0 is a cycle of
 length at least 3 and Pi for i $\in [k]$ is an 
 ear of P0 $\cup $···$\cup $Pi
    \end{definition}

    \begin{theorem}{(Whitney)}
        AgraphGis2-connected $\iff$ it has a near decomposition.Furthermore,every
 cycle of length at least 3 in a 2-connected graph G is the initial cycle in some
 ear decomposition.
    \end{theorem}

% 第三部分(Chapter 3)
\section{Matching and factor}
    \begin{definition}
        A matching in a graph G is a set of non-loop edges with no shared endpoints.
 The vertices incident to the edges of a matching M are saturated by M or
 M-saturated; the others are M-unsaturated. A perfect matching in a graph is
 a matching that saturates every vertex. The size of a matching is the number
 of edges in it.
    \end{definition}

\section{HW}

\begin{problem}
 Two people play a game on a graph G, alternately choosing distinct vertices. Player 1 starts by
 choosing any vertex. Each subsequent choice must be adjacent to the preceding choice (of the other
 player). Thus together they follow a path. The last player able to move wins.
 Prove that Player 2 has a winning strategy if G has a perfect matching, and otherwise Player 1 has
 a winning strategy.
\end{problem}

\begin{solution}
 If G has a perfect matching M, then Player 2 cannot lose if they always chooses the vertex paired
 with the preceding choice of Player 1. There is no draw in the game because the graph is finite and
 eventually somebody has no vertex to choose. Thus Player 2 can win in this way.
 If G does not have a perfect matching, let M be a maximum matching of G. Player 1 starts by
 choosing a vertex u that is M-unsaturated. From now on, when it is Player 2's turn there is no
 M-unsaturated vertex for them to choose because otherwise all the vertices that have been chosen
 form an M-augmenting path, implying that M is not maximum. So Player 2 is always forced to
 choose an M-saturated vertex, and Player 1 cannot lose if they always chooses the vertex paired
 with the preceding choice of Player 2. Similarly, Player 1 wins because there is no draw.
\end{solution}

% \subsection{hw1}
%     \begin{problem}
%  Prove that no bipartite graph contains an odd cycle    
%     \end{problem}
    
%     \begin{solution}
%     Suppose that A, B-bigraph G contains an odd cycle C = (v1, v2, . . . , v 2k+1). Without loss of generality suppose that v1 $\in$ A. Since G is bipartite and v1 v2 $\in$ E(G), v2 $\in$ B. Similarly, v2 v3 $\in$ E(G)
% and thus v3 $\in$ A, and so on. So v2k+1 $\in$ A. However v1 v2k+1 $\in$ E(G) and yet v1, v 2k+1 $\in$ A, a
% contradiction to the hypothesis that G is an A, B-bigraph.

%     \end{solution}
    
% \subsection{hw2}
%     \begin{problem}
%         Given a simple graph G, $\bar{G}$ is the simple graph with vertex set V(G) such that for any distinct
%  vertices u,v $\in$ V (G), u,v are adjacent in $\bar{G}$ if and only if u,v are not adjacent in G.
%  Prove that at least one of G and $\bar{G}$ is connected.
%     \end{problem}

%     \begin{solution}
%         If G is connected then we are done, so suppose that G is disconnected and has multiple components.
% We aim to prove that any pair of vertices x, y are connected in $\bar{G}$. If x, y belong to different
% components in G, then xy $\notin$ E(G) and xy $\in$ E($\bar{G}$). If x, y belong to the same component C in G,
% then fix a vertex v in a different component in G. Both x, y are adjacent to v in $\bar{G}$, thus x, y are
% connected in $\bar{G}$.

%     \end{solution}

% \subsection{hw3}

% \subsection{hw4}

% \subsection{hw5}

% \subsection{hw6}

\end{CJK*}
\end{document}
