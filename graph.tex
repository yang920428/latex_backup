% !TEX root = graph.tex

\documentclass[twocolumn]{article}
\usepackage{graphicx} % Required for inserting images
\usepackage{CJKutf8}
% \usepackage{amsthm}
\usepackage{mdframed}
\usepackage{multicol} % 用於分欄
\usepackage{geometry} % 用於調整頁面尺寸和邊距
\usepackage{amsmath}
\usepackage{titlesec}

\newtheorem{example}{example}             % 整体编号
\newtheorem{algorithm}{algorithm}
\newtheorem{theorem}{theorem}[section]  % 按 section 编号
\newtheorem{definition}{definition}
\newtheorem{axiom}{axiom}
\newtheorem{property}{property}
\newtheorem{proposition}{proposition}
\newtheorem{lemma}{lemma}
\newtheorem{corollary}{corollary}
\newtheorem{remark}{remark}
\newtheorem{condition}{condition}
\newtheorem{conclusion}{conclusion}
\newtheorem{assumption}{assumption}
\newtheorem{formula}{formula}
\newtheorem{problem}{problem}
\newtheorem{solution}{solution}

\linespread{0.5}
% \setlength{\parskip}{0.001em}

\setlength{\topsep}{0.1em} % 環境之上和之下的距離
\setlength{\partopsep}{0pt} % 環境和上下文之間的額外距離
\setlength{\itemsep}{0.5em} % 項目之間的距離
\setlength{\parsep}{0pt} % 段落之間的距離

\titlespacing*{\section}{0pt}{0pt}{0pt}

% 設定頁面尺寸和邊距
\geometry{
    top=0.5cm,
    bottom=0.5cm,
    left=0.5cm,
    right=0.5cm
}


% 自定義 "Definition" 環境
% \newmdtheoremenv{definition}{Definition}
% \theoremstyle{mystyle}

\title{}
% \author{}
% \date{}

\begin{document}
\begin{CJK*}{UTF8}{bkai}

% \maketitle

% 第一部分(Chapter 1)
\section{Fundamentals}

    % \switchcolumn[0]
    % 左欄內容
    % \newtheorem{definition}{def}
    \begin{definition}
        The degree of vertex u, denoted deg(u), is the number of edges that are
 incident to u, except that each loop at u counts twice.
    \end{definition}

    \begin{definition}
    The size of G is the number of its edges, usually denoted as e(G) or |E(G)|.
    \end{definition}
    
    \begin{definition}
        The order of a graph G is the number of its vertices, usually denoted as n,
 n(G), or |V (G)|.
    \end{definition}
    
    \begin{formula}
    \[
    \sum_{v \in V(G)} \deg(v) = 2 |E(G)|
    \]
    \end{formula}

    \begin{definition}
        A complete graph Kn is a graph of order n s.t. every pair of vertices are
 adjacent.
    \end{definition}

    \begin{definition}
        A path Pn is a graph s.t. V(Pn) = {v1,v2,...,vn} and
 E(Pn) = $\{v_{1}v_{2},v_{2}v_{3},...,v_{n−2}v_{n−1},v_{n−1}v_{n}\}$
    \end{definition}

    \begin{definition}
        A cycle Cn is a graph obtained from Pn by adding edge vn v1.
    \end{definition}

    \begin{definition}
     A graph H is a subgraph of G if H can be obtained from G by deleting any
 number of vertices (and their incident edges) and any number of remaining
 edges. We say G contains H, H $\subseteq$ G, or just G has H.        
    \end{definition}
    
    \begin{definition}
        Let T $\subseteq$ V(G). A graph H is a induced subgraph of G if H can be obtained
 from G by deleting all vertices not in T. We write H = G[T], or H is the
 subgraph of G induced by T.       
    \end{definition}

    \begin{definition}
         Intuitively if graphs G and H are equivalent except for names, we say G is
 isomorphic to H, there is an isomorphism from H to G, or G and H are
 isomorphic. We denote it as G $\cong$ H
    \end{definition}

    \begin{theorem}{(Konig's Theorem on bipartite graphs)}
        A graph G is bipartite $\iff$ it has no odd cycle.
    \end{theorem}

    \begin{definition}
        A walk is a list v0,e1,v1,e2,v2,...,ek,vk of vertices and edges s.t. for each
 1 $\leq$i$\leq$k, the edge ei has endpoints vi−1 and vi.
    \end{definition}

    \begin{definition}
        A trail is a walk without repeated edges.
    \end{definition}

    \begin{definition}
         A path is a trail without repeated vertices.
    \end{definition}

    \begin{definition}
         A u,v-walk, u,v-trail, or u,v-path is a walk, trail, or path that starts and ends
 at vertices u and v, respectively. u can be identical to v. The length of a walk
 is the number of edges it contains.
    \end{definition}

    \begin{lemma}
        A walk W is a path $\iff$ it does not have repeated vertices.
    \end{lemma}

    % \switchcolumn

    \begin{lemma}
        Every u,v-walk W contains a u,v-path.
    \end{lemma}

    \begin{definition}
         A graph G is connected if it has a u,v-path(walk) for every u,v $\in$ V(G).
 Otherwise it is disconnected. If G has a u,v-path(walk) then u is connected to
 v.
    \end{definition}

    \begin{definition}
        The connection relation between vertices is transitive, reflexive, and
 symmetric. So it is an equivalence relation and defines equivalence classes,
 which we call connected components.
    \end{definition}

    \begin{lemma}
         Every graph with n vertices and k edges has at least n −k components, i.e.
 $e(G) +c(G) \geq n(G)$.
    \end{lemma}

    \begin{definition}
         c(G) = number of components in G
    \end{definition}

    \begin{lemma}
         Every closed odd walk W contains an odd cycle.
    \end{lemma}

    \begin{definition}
        A graph is Eulerian if it has a closed trail containing all edges. We call a
 closed trail a circuit if we do not specify the first vertex but keep the list in
 cyclic order. An Eulerian circuit or Eulerian trail is a circuit or trail
 containing all the edges.
    \end{definition}

    \begin{definition}
        An even graph is a graph with vertex degrees all even. A vertex is odd/even
 when its degree is odd/even.
    \end{definition}

    \begin{lemma}
        If every vertex of a graph G has a degree at least 2, then G contains a cycle.
    \end{lemma}

    \begin{definition}
         A component or graph is trivial if it has no edges; otherwise it is nontrivial.
 An isolated vertex is a vertex of degree 0.
    \end{definition}

    \begin{theorem}{(Euler's Theorem)}
        A graph G is Eulerian $\iff$ it has at most one nontrivial component and its
 vertices all have even degree.
    \end{theorem}
    
    \begin{definition}
        A decomposition of a graph is a list of subgraphs s.t. every edge appears in
 exactly one subgraph in the list.
    \end{definition}

    \begin{definition}
        A tournament is a digraph where there is an edge between every pair of
 vertices.
    \end{definition}

    \begin{definition}
         A king in a digraph is a vertex from which every vertex is reachable by a path
 of length at most 2.
    \end{definition}

    \begin{proposition}
        Every tournament has a king.
    \end{proposition}

    \begin{definition}
         A sequence d = (d1,d2,...,dn) is graphic if it is the degree sequence of some
 simple graph.
    \end{definition}

    \begin{theorem}{(Havel-Hakimi)}
         For n >1, an integer list d of size n is graphic $\iff$ $d\prime$ is graphic, where $d\prime$ is
 obtained from d by deleting its largest element $\Delta$ and subtracting 1 from its $\Delta$
 next largest elements. The only 1-element graphic sequence is d = (0).
    \end{theorem}

    \begin{definition}
         A cut-edge or cut-vertex of a graph is an edge or vertex whose deletion
 increases the number of components.
    \end{definition}

    \begin{theorem}
        An edge e is a cut-edge $\iff$ it belongs to no cycle.
    \end{theorem}

    % 右欄內容
    
    

% \vspace{2cm} % 分隔不同章節

% 第二部分(Chapter 2)
\section{Tree}
% \begin{multicol}{2} % 第二部分分欄
    % \switchcolumn[0]

    \begin{theorem}{(Cayley's Formula,)}
         For a set S $\subseteq$ N of size n, there are $n^{n-2}$ trees with vertex set S.
    \end{theorem}

    \begin{lemma}
        Vertex v is a leaf of a tree T $\iff$ v is missing in $T's$ Prufer code f(T).
    \end{lemma}

    \begin{lemma}
        Let S $\subseteq$ N and n=|S|. Prufer algorithm f : \{trees with vertex set S\} → $S^{n-2}$
 is a bijection and thus Cayley's Formula holds.
    \end{lemma}    

    \begin{definition}
         $\tau$(G): the number of spanning trees of connected graph G
    \end{definition}

    \begin{definition}
        M(x|y) is the matrix obtained from M by deleting row (set) x and deleting
 column (set) y.

 M(x|y] is the matrix obtained from M by deleting row (set) x and keeping
 column (set) y.

 M[x|y) is the matrix obtained from M by keeping row (set) x and deleting
 column (set) y.

 M[x|y] is the matrix obtained from M by keeping row (set) x and keeping
 column (set) y.

 [•| means keeping all rows
 
 |•] means keeping all columns
    \end{definition}

    % \switchcolumn

    \begin{theorem}{(Matrix Tree theorem)}
        Given a loopless graph G with vertex set v1,...,vn, let ai,j be the number of
 edges with endpoints vi and vj. Let L be the matrix in which Li,j is -ai,j if
 i $\neq$ j and is deg(vi) when i = j. Then
 $\tau(G) = (-1)^{s+t}detL(s|t)$
 for any row s and any column t.
    \end{theorem}

    \begin{lemma}{(Binet-Cauchy Formula)}
        Let \( C_{n \times n} = A_{n \times m} B_{m \times n} \). Then
        \[
        \det C = \sum_{S \in \binom{[m]}{n}} \det A[\bullet |S] \det B[S| \bullet]
        \]

        where $\binom{[m]}{n}$ denote the set of all n-element subsets of [m].
    \end{lemma}

    \begin{theorem}{(MST-preservingTheorem)}
        Let G =(V,E) be a connected graph and w : E(G) → R be a weight function.
 Let A $\subset$E(G) be contained in some minimum spanning tree for G, let
 [S, V -S] be a cut s.t. no edge in A crosses [S,V -S]. Suppose that uv is a
 cheapest edge crossing [S,V - S] , then A $\cup$ \{uv\} is contained in some
 minimum spanning tree for G.
    \end{theorem}

    \begin{definition}
         A minimum connected spanning subgraph (MCSS) of a weighted connected
 graph G is a connected spanning subgraph with minimum total edge weight.
    \end{definition}

    \begin{lemma}
        If all edge weights of a connected graph G are positive, then MCSS of G is
 equivalent to MST of G.
    \end{lemma}

    \begin{lemma}
        If all edge weights of a connected graph G are nonnegative, then every MCSS
 of G contains an MST of G.
    \end{lemma}

    \begin{lemma}
         Every MCSS of G contains all edges with negative weight.
    \end{lemma}
    % 左欄內容
    % \begin{definition}
    %     這是第二章左欄的定義示例。
    % \end{definition}
    % 這是第二章的左欄內容示例。

    % \switchcolumn
    % % 右欄內容
    % 這是第二章的右欄內容示例。
    % \begin{definition}
    %     這是第二章右欄的另一個定義示例。
    % \end{definition}
% \end{multicol}
% \vspace{2cm} % 分隔不同章節

% 第三部分(Chapter 3)
\section{Matching and factor}
    \begin{definition}
        A matching in a graph G is a set of non-loop edges with no shared endpoints.
 The vertices incident to the edges of a matching M are saturated by M or
 M-saturated; the others are M-unsaturated. A perfect matching in a graph is
 a matching that saturates every vertex. The size of a matching is the number
 of edges in it.
    \end{definition}

    \begin{definition}
        A maximal matching in a graph is a matching that cannot be enlarged by
 adding an edge. A maximum matching is a matching of maximum size among
 all matchings in a graph.
    \end{definition}

    \begin{definition}
        Given a matching M, an M-alternating path is a path that alternates between
 edges in M and edges not in M. An M-alternating path whose endpoints are
 unsaturated by M is an M-augmenting path.
    \end{definition}

    \begin{definition}
        Given two matching M,M', the symmetric difference of M,M' is
 (M -M') $\cup$(M'-M), i.e., the edges in M but not M' and the edges in M'
 but not M.
    \end{definition}

    \begin{lemma}
        Every (nontrivial) component of the symmetric difference of two matchings
 M,M' is a path or an even cycle.
    \end{lemma}

    \begin{theorem}{(Berge)}
        A matching M in a graph G is maximum $\iff$ G has no M-augmenting path.
    \end{theorem}

    \begin{theorem}{ (Hall's Theorem, P. Hall 1935)}
         An X,Y-bigraph G has a matching that saturates X $\iff$ |N(S)| $\geq$ |S| for all
 S $\subseteq$ X.
    \end{theorem}

    \begin{corollary}{(Marriage Theorem)}
        An X,Y-bigraph G, where |X| = |Y|, has a perfect matching $\iff$ |N(S)| $\geq$ |S|
 for all S $\subseteq$ X.
    \end{corollary}

    \begin{definition}
        System of Distinct Representatives (SDR)
 There are n clubs in XXX University. An SDR is a set of distinct club
 representatives, one for each club.
    \end{definition}

    \begin{theorem}
        An SDR exists $\iff$ for every set S of clubs, the size of the unions of their
 members is at least |S|.
    \end{theorem}

    \begin{definition}
        An n$\times$n 0-1 matrix A does not contain an n$\times$n permutation matrix $\iff$ A
 has an r $\times$s all-zeros submatrix with r +s > n.
    \end{definition}

    \begin{corollary}
         For k > 0, every k-regular bipartite graph has a perfect matching.
    \end{corollary}

    \begin{definition}
         A decomposition of a graph is a list of subgraphs s.t. every edge appears in
 exactly one subgraph in the list.
    \end{definition}

    \begin{corollary}
        A k-regular bipartite graph G can be decomposed into k perfect matchings.
    \end{corollary}

    \begin{definition}
        A vertex cover of a graph G is a set Q $\subseteq$ V(G) that contains at least one
 endpoint of every edge. The vertices in Q cover E(G).
 Trivially, V (G) is a vertex cover of G. The minimum cardinality of a vertex
 cover of G is denoted by $\beta(G)$.
    \end{definition}

    \begin{definition}
        $\alpha$(G): maximum size of independent set

 $\alpha$'(G): maximum size of matching
 $\beta$(G): minimum size of vertex cover
 $\beta$'(G): minimum size of edge cover
    \end{definition}

    \begin{lemma}
        In a graph G, S $\subseteq$ V(G) is an independent set $\iff$ V(G)-S is a vertex cover,
 and hence $\alpha$(G)+$\beta$(G) = n(G).
    \end{lemma}

    \begin{theorem}{(Konig 1931, Egervary 1931)}
        If G is a bipartite graph, then the maximum size of a matching in G is equal
 to the minimum size of a vertex cover of G, i.e., $\alpha$'(G) = $\beta$(G).
    \end{theorem}

    \begin{corollary}
         If we find a matching and a vertex cover, both of size k, in bipartite graph G,
 then $\alpha$'(G) = $\beta$(G) = k.
    \end{corollary}

    \begin{definition}
        An edge cover of G is a set L of edges s.t. every vertex of G is incident to
 some edge of L.
 The minimum size of edge cover of G is denoted by $\beta$'(G).
    \end{definition}

    \begin{theorem}{(Gallai 1959)}
        If G is a graph without isolated vertices, then $\alpha$'(G) + $\beta$'(G) = n(G), i.e., the
 maximum size of matching and the minimum size of edge cover sum up to the
 order of G.
    \end{theorem}

    \begin{corollary}{(Konig 1916)}
        If G is a bipartite graph with no isolated vertices, then $\alpha$(G) = $\beta$'(G), i.e., the
 size of maximum independent set is equal to the size of minimum edge cover.
    \end{corollary}

    \begin{corollary}
        If we find an independent set and an edge cover, both of size k, in bipartite
 graph G without isolated vertices, then $\alpha$(G) = $\beta$'(G) = k.
    \end{corollary}

    \begin{definition}
        A factor of a graph G is a spanning subgraph of G, i.e., with vertex set V (G).
 A k-factor is a spanning k-regular subgraph. An odd component of a graph is
 a component of odd order; the number of odd component of G is denoted o(G).

 1-factor are almost same as perfect matching:

 1-factor: subgraph
 
 perfect matching: set of edges
    \end{definition}

\section{HW}

\begin{problem}
 Two people play a game on a graph G, alternately choosing distinct vertices. Player 1 starts by
 choosing any vertex. Each subsequent choice must be adjacent to the preceding choice (of the other
 player). Thus together they follow a path. The last player able to move wins.
 Prove that Player 2 has a winning strategy if G has a perfect matching, and otherwise Player 1 has
 a winning strategy.
\end{problem}

\begin{solution}
 If G has a perfect matching M, then Player 2 cannot lose if they always chooses the vertex paired
 with the preceding choice of Player 1. There is no draw in the game because the graph is finite and
 eventually somebody has no vertex to choose. Thus Player 2 can win in this way.
 If G does not have a perfect matching, let M be a maximum matching of G. Player 1 starts by
 choosing a vertex u that is M-unsaturated. From now on, when it is Player 2's turn there is no
 M-unsaturated vertex for them to choose because otherwise all the vertices that have been chosen
 form an M-augmenting path, implying that M is not maximum. So Player 2 is always forced to
 choose an M-saturated vertex, and Player 1 cannot lose if they always chooses the vertex paired
 with the preceding choice of Player 2. Similarly, Player 1 wins because there is no draw.
\end{solution}

\begin{problem}
     Let G be an X,Y-bigraph such that |N(S)| > |S| whenever $\emptyset$ $\notin$ S $\subset $ X(S $\neq$ X). Prove that every edge of
 G belongs to some matching that saturates X.
\end{problem}

\begin{solution}
 It suffices to show that for every edge xy where x $\in$ X,y $\in$ Y, bipartite graph G' = G-x-y has
 a matching that saturates X' = X - {x}. Below we check that the Hall's condition holds for G'.
 For every vertex set S $\subset$ X',
 \[
 \   |N_{G'}(S)| \geq |N_{G}(S)| -1 \geq |S|
 \]
 Therefore the Hall's condition holds for G', and the proof is completed.
\end{solution}


\begin{problem}
Use the Konig-Egervary Theorem to prove that every bipartite graph \( G \) has a matching of size at least \( \frac{e(G)}{\Delta(G)} \), where \( \Delta(G) \) is the maximum degree of all vertices in \( G \). Use this to conclude that every subgraph of \( K_{m,m} \) with more than \( (k - 1)m \) edges has a matching of size at least \( k \).
\end{problem}

\begin{solution}
By the Konig-Egervary Theorem, the size of maximum matching is equal to the size of minimum vertex cover. Each vertex in the minimum vertex cover is incident to at most \( \Delta(G) \) edges, so the size of minimum vertex cover

\[
\beta(G) \geq \frac{e(G)}{\Delta(G)}.
\]

Thus the size of maximum matching is at least \( \frac{e(G)}{\Delta(G)} \).

For any subgraph \( G \) of \( K_{m,m} \) with more than \( (k - 1)m \) edges, \( e(G) > (k - 1)m \) and \( \Delta(G) \leq m \). Thus

\[
\alpha'(G) = \beta(G) \geq \frac{e(G)}{\Delta(G)} > k - 1
\]

and \( \alpha'(G) \geq k \).
\end{solution}

\begin{problem}
     Prove that if man x is paired with woman a in some stable matching M, then a does not reject x
 in the Gale-Shapley Proposal Algorithm with men proposing.
\end{problem}

\begin{solution}
     For clarity and brevity we use ax > bx to denote that for x, a ranks higher than b.
 We prove by contradiction. In a certain run of the algorithm that produces stable matching M',
 let a rejecting x be the first occurrence of some woman rejecting her partner in M. This rejection
 is caused by the proposal of man y to a where ya > xa and y is paired with a in M'. y and a are
 not paired in M, and for M to be a stable matching y is paired with woman b where by > ay. In
 the same run of the algorithm, y's proposal to a implies that b rejected y. So a rejecting x is not
 the first such occurrence, because b rejecting y precedes it.
\end{solution}

\begin{problem}
    Use Cayley's Formula to prove that the graph obtained from \( K_n \) by deleting an edge has exactly \( (n - 2)n^{n - 3} \) spanning trees.

\end{problem}

\begin{solution}
By Cayley's Formula \( K_n \) has \( n^{n - 2} \) spanning trees, each with \( n - 1 \) edges. \( K_n \) has \( \binom{n}{2} \) edges. By symmetry and double counting ordered pair (edge of \( K_n \), spanning tree of \( K_n \)) each edge of \( K_n \) is in

\[
\frac{n^{n - 2} (n - 1)}{\binom{n}{2}} = 2n^{n - 3}
\]

spanning trees of \( K_n \). Thus for any specific edge \( e \) of \( K_n \) there are \( n^{n - 2} - 2n^{n - 3} = (n - 2)n^{n - 3} \) spanning trees of \( K_n \) not containing \( e \), and this is also the number of spanning trees of \( K_n - e \).
\end{solution}


% \subsection{hw1}
%     \begin{problem}
%  Prove that no bipartite graph contains an odd cycle    
%     \end{problem}
    
%     \begin{solution}
%     Suppose that A, B-bigraph G contains an odd cycle C = (v1, v2, . . . , v 2k+1). Without loss of generality suppose that v1 $\in$ A. Since G is bipartite and v1 v2 $\in$ E(G), v2 $\in$ B. Similarly, v2 v3 $\in$ E(G)
% and thus v3 $\in$ A, and so on. So v2k+1 $\in$ A. However v1 v2k+1 $\in$ E(G) and yet v1, v 2k+1 $\in$ A, a
% contradiction to the hypothesis that G is an A, B-bigraph.

%     \end{solution}
    
% \subsection{hw2}
%     \begin{problem}
%         Given a simple graph G, $\bar{G}$ is the simple graph with vertex set V(G) such that for any distinct
%  vertices u,v $\in$ V (G), u,v are adjacent in $\bar{G}$ if and only if u,v are not adjacent in G.
%  Prove that at least one of G and $\bar{G}$ is connected.
%     \end{problem}

%     \begin{solution}
%         If G is connected then we are done, so suppose that G is disconnected and has multiple components.
% We aim to prove that any pair of vertices x, y are connected in $\bar{G}$. If x, y belong to different
% components in G, then xy $\notin$ E(G) and xy $\in$ E($\bar{G}$). If x, y belong to the same component C in G,
% then fix a vertex v in a different component in G. Both x, y are adjacent to v in $\bar{G}$, thus x, y are
% connected in $\bar{G}$.

%     \end{solution}

% \subsection{hw3}

% \subsection{hw4}

% \subsection{hw5}

% \subsection{hw6}

\end{CJK*}
\end{document}
