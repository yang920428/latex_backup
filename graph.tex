% !TEX root = graph.tex

\documentclass[2pt]{article}
\usepackage{graphicx} % Required for inserting images
\usepackage{CJKutf8}
% \usepackage{amsthm}
\usepackage{mdframed}
\usepackage{paracol} % 用於分欄
\usepackage{geometry} % 用於調整頁面尺寸和邊距
\usepackage{amsmath}

\newtheorem{example}{example}             % 整体编号
\newtheorem{algorithm}{algorithm}
\newtheorem{theorem}{theorem}[section]  % 按 section 编号
\newtheorem{definition}{definition}
\newtheorem{axiom}{axiom}
\newtheorem{property}{property}
\newtheorem{proposition}{proposition}
\newtheorem{lemma}{lemma}
\newtheorem{corollary}{corollary}
\newtheorem{remark}{remark}
\newtheorem{condition}{condition}
\newtheorem{conclusion}{conclusion}
\newtheorem{assumption}{assumption}
\newtheorem{formula}{formula}

\linespread{1}
\setlength{\parskip}{0.01em}

% 設定頁面尺寸和邊距
\geometry{
    top=0.5cm,
    bottom=0.5cm,
    left=0.5cm,
    right=0.5cm
}


% 自定義 "Definition" 環境
% \newmdtheoremenv{definition}{Definition}
% \theoremstyle{mystyle}

\title{}
% \author{}
% \date{}

\begin{document}
\begin{CJK*}{UTF8}{bkai}

% \maketitle

% 第一部分(Chapter 1)
\section*{Fundamentals}
\begin{paracol}{2} % 開始分欄
    \switchcolumn[0]
    % 左欄內容
    % \newtheorem{definition}{def}
    \begin{definition}
        The degree of vertex u, denoted deg(u), is the number of edges that are
 incident to u, except that each loop at u counts twice.
    \end{definition}
    
    \begin{definition}
    The size of G is the number of its edges, usually denoted as e(G) or |E(G)|.
    \end{definition}
    
    \begin{definition}
        The order of a graph G is the number of its vertices, usually denoted as n,
 n(G), or |V (G)|.
    \end{definition}
    
    \begin{formula}
    \[
    \sum_{v \in V(G)} \deg(v) = 2 |E(G)|
    \]
    \end{formula}

    \begin{definition}
        A complete graph Kn is a graph of order n s.t. every pair of vertices are
 adjacent.
    \end{definition}

    \begin{definition}
        A path Pn is a graph s.t. V(Pn) = {v1,v2,...,vn} and
 E(Pn) = $\{v_{1}v_{2},v_{2}v_{3},...,v_{n−2}v_{n−1},v_{n−1}v_{n}\}$
    \end{definition}

    \begin{definition}
        A cycle Cn is a graph obtained from Pn by adding edge vn v1.
    \end{definition}

    \begin{definition}
     A graph H is a subgraph of G if H can be obtained from G by deleting any
 number of vertices (and their incident edges) and any number of remaining
 edges. We say G contains H, H $\subseteq$ G, or just G has H.        
    \end{definition}
    
    \begin{definition}
        Let T $\subseteq$ V(G). A graph H is a induced subgraph of G if H can be obtained
 from G by deleting all vertices not in T. We write H = G[T], or H is the
 subgraph of G induced by T.       
    \end{definition}

    \begin{definition}
         Intuitively if graphs G and H are equivalent except for names, we say G is
 isomorphic to H, there is an isomorphism from H to G, or G and H are
 isomorphic. We denote it as G $\cong$ H
    \end{definition}

    \begin{theorem}
        A graph G is bipartite $\iff$ it has no odd cycle.
    \end{theorem}

    \begin{definition}
        A walk is a list v0,e1,v1,e2,v2,...,ek,vk of vertices and edges s.t. for each
 1 $\leq$i$\leq$k, the edge ei has endpoints vi−1 and vi.
    \end{definition}

    \begin{definition}
        A trail is a walk without repeated edges.
    \end{definition}

    \begin{definition}
         A path is a trail without repeated vertices.
    \end{definition}

    \begin{definition}
         A u,v-walk, u,v-trail, or u,v-path is a walk, trail, or path that starts and ends
 at vertices u and v, respectively. u can be identical to v. The length of a walk
 is the number of edges it contains.
    \end{definition}

    \begin{lemma}
        A walk W is a path $\iff$ it does not have repeated vertices.
    \end{lemma}

    \switchcolumn

    \begin{lemma}
        Every u,v-walk W contains a u,v-path.
    \end{lemma}

    \begin{definition}
         A graph G is connected if it has a u,v-path(walk) for every u,v $\in$ V(G).
 Otherwise it is disconnected. If G has a u,v-path(walk) then u is connected to
 v.
    \end{definition}

    \begin{definition}
        The connection relation between vertices is transitive, reflexive, and
 symmetric. So it is an equivalence relation and defines equivalence classes,
 which we call connected components.
    \end{definition}

    \begin{lemma}
         Every graph with n vertices and k edges has at least n −k components, i.e.
 $e(G) +c(G) \geq n(G)$.
    \end{lemma}

    \begin{definition}
         c(G) = number of components in G
    \end{definition}

    \begin{lemma}
         Every closed odd walk W contains an odd cycle.
    \end{lemma}

    \begin{definition}
        A graph is Eulerian if it has a closed trail containing all edges. We call a
 closed trail a circuit if we do not specify the first vertex but keep the list in
 cyclic order. An Eulerian circuit or Eulerian trail is a circuit or trail
 containing all the edges.
    \end{definition}

    \begin{definition}
        An even graph is a graph with vertex degrees all even. A vertex is odd/even
 when its degree is odd/even.
    \end{definition}

    \begin{lemma}
        If every vertex of a graph G has a degree at least 2, then G contains a cycle.
    \end{lemma}

    \begin{definition}
         A component or graph is trivial if it has no edges; otherwise it is nontrivial.
 An isolated vertex is a vertex of degree 0.
    \end{definition}

    \begin{theorem}
        A graph G is Eulerian $\iff$ it has at most one nontrivial component and its
 vertices all have even degree.
    \end{theorem}
    
    \begin{definition}
        A decomposition of a graph is a list of subgraphs s.t. every edge appears in
 exactly one subgraph in the list.
    \end{definition}

    \begin{definition}
        A tournament is a digraph where there is an edge between every pair of
 vertices.
    \end{definition}

    \begin{definition}
         A king in a digraph is a vertex from which every vertex is reachable by a path
 of length at most 2.
    \end{definition}

    \begin{proposition}
        Every tournament has a king.
    \end{proposition}

    \begin{definition}
         A sequence d = (d1,d2,...,dn) is graphic if it is the degree sequence of some
 simple graph.
    \end{definition}

    \begin{theorem}
         For n >1, an integer list d of size n is graphic $\iff$ $d\prime$ is graphic, where $d\prime$ is
 obtained from d by deleting its largest element $\Delta$ and subtracting 1 from its $\Delta$
 next largest elements. The only 1-element graphic sequence is d = (0).
    \end{theorem}

    \begin{definition}
         A cut-edge or cut-vertex of a graph is an edge or vertex whose deletion
 increases the number of components.
    \end{definition}

    \begin{theorem}
        An edge e is a cut-edge $\iff$ it belongs to no cycle.
    \end{theorem}

    % 右欄內容
    
    
\end{paracol}
\vspace{2cm} % 分隔不同章節

% 第二部分(Chapter 2)
\section*{Tree}
\begin{paracol}{2} % 第二部分分欄
    \switchcolumn[0]
    % 左欄內容
    % \begin{definition}
    %     這是第二章左欄的定義示例。
    % \end{definition}
    % 這是第二章的左欄內容示例。

    % \switchcolumn
    % % 右欄內容
    % 這是第二章的右欄內容示例。
    % \begin{definition}
    %     這是第二章右欄的另一個定義示例。
    % \end{definition}
\end{paracol}
\vspace{2cm} % 分隔不同章節

% 第三部分(Chapter 3)
\section*{Matching and factor}
\begin{paracol}{2} % 第三部分分欄
    \switchcolumn[0]
    % 左欄內容
    % \begin{definition}
    %     這是第三章左欄的定義示例。
    % \end{definition}
    % 這是第三章的左欄內容示例。

    % \switchcolumn
    % % 右欄內容
    % 這是第三章的右欄內容示例。
    % \begin{definition}
    %     這是第三章右欄的另一個定義示例。
    % \end{definition}
\end{paracol}

\end{CJK*}
\end{document}
